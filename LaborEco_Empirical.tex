\documentclass[]{book}
\usepackage{lmodern}
\usepackage{amssymb,amsmath}
\usepackage{ifxetex,ifluatex}
\usepackage{fixltx2e} % provides \textsubscript
\ifnum 0\ifxetex 1\fi\ifluatex 1\fi=0 % if pdftex
  \usepackage[T1]{fontenc}
  \usepackage[utf8]{inputenc}
\else % if luatex or xelatex
  \ifxetex
    \usepackage{mathspec}
  \else
    \usepackage{fontspec}
  \fi
  \defaultfontfeatures{Ligatures=TeX,Scale=MatchLowercase}
\fi
% use upquote if available, for straight quotes in verbatim environments
\IfFileExists{upquote.sty}{\usepackage{upquote}}{}
% use microtype if available
\IfFileExists{microtype.sty}{%
\usepackage{microtype}
\UseMicrotypeSet[protrusion]{basicmath} % disable protrusion for tt fonts
}{}
\usepackage[margin=1in]{geometry}
\usepackage{hyperref}
\hypersetup{unicode=true,
            pdftitle={A Minimal Book Example},
            pdfauthor={Yihui Xie},
            pdfborder={0 0 0},
            breaklinks=true}
\urlstyle{same}  % don't use monospace font for urls
\usepackage{natbib}
\bibliographystyle{apalike}
\usepackage{color}
\usepackage{fancyvrb}
\newcommand{\VerbBar}{|}
\newcommand{\VERB}{\Verb[commandchars=\\\{\}]}
\DefineVerbatimEnvironment{Highlighting}{Verbatim}{commandchars=\\\{\}}
% Add ',fontsize=\small' for more characters per line
\usepackage{framed}
\definecolor{shadecolor}{RGB}{248,248,248}
\newenvironment{Shaded}{\begin{snugshade}}{\end{snugshade}}
\newcommand{\AlertTok}[1]{\textcolor[rgb]{0.94,0.16,0.16}{#1}}
\newcommand{\AnnotationTok}[1]{\textcolor[rgb]{0.56,0.35,0.01}{\textbf{\textit{#1}}}}
\newcommand{\AttributeTok}[1]{\textcolor[rgb]{0.77,0.63,0.00}{#1}}
\newcommand{\BaseNTok}[1]{\textcolor[rgb]{0.00,0.00,0.81}{#1}}
\newcommand{\BuiltInTok}[1]{#1}
\newcommand{\CharTok}[1]{\textcolor[rgb]{0.31,0.60,0.02}{#1}}
\newcommand{\CommentTok}[1]{\textcolor[rgb]{0.56,0.35,0.01}{\textit{#1}}}
\newcommand{\CommentVarTok}[1]{\textcolor[rgb]{0.56,0.35,0.01}{\textbf{\textit{#1}}}}
\newcommand{\ConstantTok}[1]{\textcolor[rgb]{0.00,0.00,0.00}{#1}}
\newcommand{\ControlFlowTok}[1]{\textcolor[rgb]{0.13,0.29,0.53}{\textbf{#1}}}
\newcommand{\DataTypeTok}[1]{\textcolor[rgb]{0.13,0.29,0.53}{#1}}
\newcommand{\DecValTok}[1]{\textcolor[rgb]{0.00,0.00,0.81}{#1}}
\newcommand{\DocumentationTok}[1]{\textcolor[rgb]{0.56,0.35,0.01}{\textbf{\textit{#1}}}}
\newcommand{\ErrorTok}[1]{\textcolor[rgb]{0.64,0.00,0.00}{\textbf{#1}}}
\newcommand{\ExtensionTok}[1]{#1}
\newcommand{\FloatTok}[1]{\textcolor[rgb]{0.00,0.00,0.81}{#1}}
\newcommand{\FunctionTok}[1]{\textcolor[rgb]{0.00,0.00,0.00}{#1}}
\newcommand{\ImportTok}[1]{#1}
\newcommand{\InformationTok}[1]{\textcolor[rgb]{0.56,0.35,0.01}{\textbf{\textit{#1}}}}
\newcommand{\KeywordTok}[1]{\textcolor[rgb]{0.13,0.29,0.53}{\textbf{#1}}}
\newcommand{\NormalTok}[1]{#1}
\newcommand{\OperatorTok}[1]{\textcolor[rgb]{0.81,0.36,0.00}{\textbf{#1}}}
\newcommand{\OtherTok}[1]{\textcolor[rgb]{0.56,0.35,0.01}{#1}}
\newcommand{\PreprocessorTok}[1]{\textcolor[rgb]{0.56,0.35,0.01}{\textit{#1}}}
\newcommand{\RegionMarkerTok}[1]{#1}
\newcommand{\SpecialCharTok}[1]{\textcolor[rgb]{0.00,0.00,0.00}{#1}}
\newcommand{\SpecialStringTok}[1]{\textcolor[rgb]{0.31,0.60,0.02}{#1}}
\newcommand{\StringTok}[1]{\textcolor[rgb]{0.31,0.60,0.02}{#1}}
\newcommand{\VariableTok}[1]{\textcolor[rgb]{0.00,0.00,0.00}{#1}}
\newcommand{\VerbatimStringTok}[1]{\textcolor[rgb]{0.31,0.60,0.02}{#1}}
\newcommand{\WarningTok}[1]{\textcolor[rgb]{0.56,0.35,0.01}{\textbf{\textit{#1}}}}
\usepackage{longtable,booktabs}
\usepackage{graphicx,grffile}
\makeatletter
\def\maxwidth{\ifdim\Gin@nat@width>\linewidth\linewidth\else\Gin@nat@width\fi}
\def\maxheight{\ifdim\Gin@nat@height>\textheight\textheight\else\Gin@nat@height\fi}
\makeatother
% Scale images if necessary, so that they will not overflow the page
% margins by default, and it is still possible to overwrite the defaults
% using explicit options in \includegraphics[width, height, ...]{}
\setkeys{Gin}{width=\maxwidth,height=\maxheight,keepaspectratio}
\IfFileExists{parskip.sty}{%
\usepackage{parskip}
}{% else
\setlength{\parindent}{0pt}
\setlength{\parskip}{6pt plus 2pt minus 1pt}
}
\setlength{\emergencystretch}{3em}  % prevent overfull lines
\providecommand{\tightlist}{%
  \setlength{\itemsep}{0pt}\setlength{\parskip}{0pt}}
\setcounter{secnumdepth}{5}
% Redefines (sub)paragraphs to behave more like sections
\ifx\paragraph\undefined\else
\let\oldparagraph\paragraph
\renewcommand{\paragraph}[1]{\oldparagraph{#1}\mbox{}}
\fi
\ifx\subparagraph\undefined\else
\let\oldsubparagraph\subparagraph
\renewcommand{\subparagraph}[1]{\oldsubparagraph{#1}\mbox{}}
\fi

%%% Use protect on footnotes to avoid problems with footnotes in titles
\let\rmarkdownfootnote\footnote%
\def\footnote{\protect\rmarkdownfootnote}

%%% Change title format to be more compact
\usepackage{titling}

% Create subtitle command for use in maketitle
\providecommand{\subtitle}[1]{
  \posttitle{
    \begin{center}\large#1\end{center}
    }
}

\setlength{\droptitle}{-2em}

  \title{A Minimal Book Example}
    \pretitle{\vspace{\droptitle}\centering\huge}
  \posttitle{\par}
    \author{Yihui Xie}
    \preauthor{\centering\large\emph}
  \postauthor{\par}
      \predate{\centering\large\emph}
  \postdate{\par}
    \date{2019-04-19}

\usepackage{booktabs}

\begin{document}
\maketitle

{
\setcounter{tocdepth}{1}
\tableofcontents
}
\hypertarget{prerequisites}{%
\chapter{Prerequisites}\label{prerequisites}}

This is a \emph{sample} book written in \textbf{Markdown}. You can use anything that Pandoc's Markdown supports, e.g., a math equation \(a^2 + b^2 = c^2\).

The \textbf{bookdown} package can be installed from CRAN or Github:

\begin{Shaded}
\begin{Highlighting}[]
\KeywordTok{install.packages}\NormalTok{(}\StringTok{"bookdown"}\NormalTok{)}
\CommentTok{# or the development version}
\CommentTok{# devtools::install_github("rstudio/bookdown")}
\end{Highlighting}
\end{Shaded}

Remember each Rmd file contains one and only one chapter, and a chapter is defined by the first-level heading \texttt{\#}.

To compile this example to PDF, you need XeLaTeX. You are recommended to install TinyTeX (which includes XeLaTeX): \url{https://yihui.name/tinytex/}.

\hypertarget{intro}{%
\chapter{Basic\_knowledge}\label{intro}}

\hypertarget{how-to-write-a-empirical-paper}{%
\section{How to write a empirical paper}\label{how-to-write-a-empirical-paper}}

As a mindset, a good empirical paper rely on topics, method and writing skills.

Basically, there are seven parts in an empirical paper. Let's see the ordered list below:

\begin{enumerate}
\def\labelenumi{\arabic{enumi}.}
\tightlist
\item
  Abstract
\item
  Introduction
\item
  Conceptual (or Theoretical) Framework
\item
  Econometric Models and Estimation Methods
\item
  The Data
\item
  Results(\& Robustness check)
\item
  Conclusions
\end{enumerate}

\hypertarget{abstract}{%
\subsection{Abstract}\label{abstract}}

The capstone of your paper! Take care of it like your first and only baby because it really dominates if the reader will go into the rest of content.

\hypertarget{introduction}{%
\subsection{Introduction}\label{introduction}}

there are a few things that should appear in the introduction.

\begin{itemize}
\item
  States the basic objective and justify why that is important.
\item
  Entails a literature review in usual, which indicates what has been done and how previous work can be improved upon. But, be humble and polite in order not to hurt someone else.
\item
  Most Researcher will summarize what is discovered in this paper. This can be a useful way to grab the audience's attention.
\end{itemize}

\hypertarget{conceptual-or-theoretical-framework}{%
\subsection{Conceptual or Theoretical Framework}\label{conceptual-or-theoretical-framework}}

\begin{enumerate}
\def\labelenumi{\arabic{enumi}.}
\item
  Often, there is no need to write down an economic theory unless you build a new one.
\item
  We may spend the effort on specify the intuition and factors(or variables) that need to be controled in the model. It will bring you extra benefit on the model section because the reader may already know the variables you use.
\end{enumerate}

\hypertarget{econometric-models-and-estimation-methods}{%
\subsection{Econometric Models and Estimation Methods}\label{econometric-models-and-estimation-methods}}

\begin{enumerate}
\def\labelenumi{\arabic{enumi}.}
\item
  Describe the general approach you used to answering the question.
\item
  Writing the equations with error term will be helpful to discuss further on whether the assumption is satisified or not.
\item
  Specifying the model clearly, and then discuss the estimation methods.
\end{enumerate}

\begin{quote}
It sounds like the relationship between causality equation and real model equation. It's not talking about the estimation result table.
\end{quote}

\begin{quote}
Any assumption in the estimation method should be justified. For example, there are three assumption in IV, including the weak IV test. Don't tell the audience you want to do with IV, and jump to the causality interpretation in a blink! You even yet explain it is suitable or not\ldots{}..
\end{quote}

\begin{enumerate}
\def\labelenumi{\arabic{enumi}.}
\setcounter{enumi}{3}
\item
  Show the model(equation) in LaTeX, including the key explanatory variable(解釋變數) and controlled variable(控制變數)
\item
  Do the functional form decisions, or the robustness check.
\end{enumerate}

\begin{quote}
some variables should appear in log form or in levels or square? (For example, pre-processing or capturing a diminishing effect)
\end{quote}

\hypertarget{the-data}{%
\subsection{The Data}\label{the-data}}

\begin{quote}
Honestly, I am not sure if the Data section should be mentioned before entering the model or not?
\end{quote}

\begin{itemize}
\item
  Describe the data carefully so that the other researchers may reproduce your research.

  \begin{itemize}
  \item
    Data accessible?
  \item
    all applicable public data sources should be included in the appendix
  \item
    a copy of questionnaire of your survey should be presented in the appendix
  \item
    a table of summary statitics, such as maximum, minimum, mean, sd, distribution of each variable.
  \item
    the unit(單位) of the variable.
  \item
    Are they cross-sectional, time series, pooled cross sections, or panel data?
  \item
    the quantity of your observations.
  \end{itemize}
\end{itemize}

\hypertarget{results-robustness-check}{%
\subsection{Results(\& Robustness check)}\label{results-robustness-check}}

\begin{quote}
the results should include your estimates of \textbf{any} models formulated in the model section? So, it may be more than 1 model in whole paper?
\end{quote}

\begin{itemize}
\item
  the most important thing is to discuss the interpretation and strength of your empirical result.

  \begin{itemize}
  \item
    Does the coefficients get the expected sign? (or it may be a method or data problem)
  \item
    Is the coeffiecients significant in statistics?
  \item
    Describe the magnitude of the coefficients on your major key explantory variable.
  \end{itemize}
\end{itemize}

\hypertarget{conclusions}{%
\subsection{Conclusions}\label{conclusions}}

\begin{itemize}
\item
  Be careful on your conclusion writing because it's also the section your audience may start with.
\item
  You may summarize what you learn, also, discuss caveats(注意事項,告誡) to the conclusion drawn.
\item
  You may also mention the suggestion of future research.
\end{itemize}

\hypertarget{one-more-thing}{%
\subsection{One more thing}\label{one-more-thing}}

To pose a question, you need to locate your area first.

There are several source for you to locate specific paper once your had decided your research area(研究主題)

\begin{quote}
研究主題\textgreater{}\textgreater{}研究子領域\textgreater{}\textgreater{}論文題目
\end{quote}

\begin{itemize}
\item
  EconLit
\item
  The Social Sciences Citation Index(SSCI)
\item
  Google scholar(set the ranking(引用次數) and the publication year as your clue)
\end{itemize}

\hypertarget{on-the-dataset}{%
\subsubsection{On the dataset}\label{on-the-dataset}}

\begin{enumerate}
\def\labelenumi{\arabic{enumi}.}
\tightlist
\item
  Always do the data exploring (be specific, inspecting, cleaning, and summarizing) at first, and decide how to handle the missing value.
\end{enumerate}

\begin{quote}
Usually, we set any numerical codes(or character) for missing value.
\end{quote}

\begin{longtable}[]{@{}rrr@{}}
\toprule
a & b & c\tabularnewline
\midrule
\endhead
1 & 2 & 3\tabularnewline
2 & 3 & 4\tabularnewline
\bottomrule
\end{longtable}

\(f(x) = x^2\)

\[
\beta^1_1\\
\alpha^2_2\\
\]

\hypertarget{causality-and-selection-bias}{%
\chapter{Causality and Selection Bias}\label{causality-and-selection-bias}}

\hypertarget{the-causal-if-then-question}{%
\subsection{The causal if-then question}\label{the-causal-if-then-question}}

在實證研究要回答一個因果關係是不容易的一件事,縱使是一個直覺上非常簡單的若不小心也會產生誤導的結果。只能說,想要捕捉真實世界的樣貌並不是件容易的事,所以我們需要學習各種計量方法。

來看一個簡單的例子。

Do hospital make people healthier?

\begin{longtable}[]{@{}rrrr@{}}
\toprule
Group & Sample size & Mean Health Status & sd.\tabularnewline
\midrule
\endhead
Hospital & 7774 & 3.21 & 0.014\tabularnewline
No Hospital & 90.049 & 3.93 & 0.003\tabularnewline
\bottomrule
\end{longtable}

怎麼看起來有去醫院看醫生的人,健康狀況反而不如沒有去看醫生的人呢?難道醫療的效果是負的嗎?

其實,不能夠忽略會去醫院的人,本身的健康狀況肯定是比較差的。若我們沒有先控制住 \textbf{有去醫院的那群人},以及 \textbf{沒有去醫院的那群人},本質上的不同的話,我們就可能會做出令人匪夷所思的判斷\ldots{}

所以我們不應該由敘述統計量的帳面數字,就將其詮釋為因果關係的大小、方向等等。

\hypertarget{potential-outcome-framework}{%
\subsection{Potential Outcome Framework}\label{potential-outcome-framework}}

自1990年代開始,計量經濟學對因果關係的探究,有受到醫學實驗領域的影響,總希望能夠在現實生活中製造出一個很像實驗室真空的環境來捕捉因果關係(這也是實驗經濟學最主要的核心精神,背後比較廣泛的概念稱為Randomized Controlled Trial, RCT)。

受限於RCT的成本較高,有些可能涉及研究倫理道德議題,因此若是要用資料(Data)來下手的話,也是會想要設計成彷彿大自然冥冥之中安排好的經歷一樣,我們稱做Natural Experiment。

所有的實驗都有實驗組(treatment group),以及對照組(controlled group)。而實驗前本身就會想要設定兩組樣本的特性是一樣的,也就是俗稱的立足點相同。

但是在Natural Experiment裡面,舉例來說 \textbf{上大學對薪資的影響},同一個樣本,我永遠只能知道有上大學後現在的他,但是我就是看不到 \textbf{如果} 沒有上大學的他,表現又是如何。

\textbf{可觀察到的結果永遠只有一個}

以數學表示:

\hypertarget{potential-outcome}{%
\subsubsection{Potential Outcome}\label{potential-outcome}}

\[
Potnetial\ outcome =
\begin{cases}
Y_{1i} & if & D_i=1\\
Y_{0i} & if & D_i=0\\
\end{cases}
\]
\(D_i = 1\) 表示有接受Treatment.

\hypertarget{observed-outcome}{%
\subsubsection{Observed Outcome}\label{observed-outcome}}

i = 1則=\(Y_{1i}\)。反之,i = 0則=\(Y_{0i}\)。只有一種結果。

\[
Observed\ outcome =
Y_i = Y_{1i}D_i + Y_{0i}(1-D_i)
\]
\#\#\#\# Counterfactual outcome

對於有接受treatment,像是有上大學的小明,沒有上大學的他,是無法觀察到的,我們就稱 \textbf{沒有上大學的他} 為Counterfactual outcome。

\subsection{小結論}

由obseved outcome簡單化減:

綜合以上,因為potential outcome,我們就是無法兩個都看到。因此在可視得結果(observed outcome)下,學習進階的計量技巧,拆解出我們要的因果關係部分。

\hypertarget{causal-effect-for-an-individual}{%
\subsection{Causal Effect for an Individual}\label{causal-effect-for-an-individual}}

\hypertarget{regression}{%
\chapter{Regression}\label{regression}}

We describe our methods in this chapter.

\hypertarget{iv}{%
\chapter{IV}\label{iv}}

resource:

\begin{enumerate}
\def\labelenumi{\arabic{enumi}.}
\item
  worksheet from Dr.Hsiu-Fen Hsu
\item
  \href{https://bookdown.org/tpemartin/econometric_analysis/}{econometric\_analysis from NTPUEcon graduate school by Dr.Mau-Ting Lin}
\item
  \href{https://bookdown.org/ccolonescu/RPoE4/random-regressors.html}{PoE with R ch10}
\end{enumerate}

\hypertarget{final-words}{%
\chapter{Final Words}\label{final-words}}

We have finished a nice book.

\bibliography{book.bib,packages.bib}


\end{document}
